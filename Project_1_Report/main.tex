    % --------------------------------------------------------------
% This is all preamble stuff that you don't have to worry about.
% Head down to where it says "Start here"
% --------------------------------------------------------------
 
\documentclass[12pt]{article}
 
\usepackage[margin=1in]{geometry} 
\usepackage{amsmath,amsthm,amssymb}
\usepackage{fancyhdr}
\usepackage{graphicx}
\usepackage[
backend=biber,
style=ieee,
]{biblatex}
\addbibresource{bibliography.bib}
 
\newcommand{\N}{\mathbb{N}}
\newcommand{\Z}{\mathbb{Z}}
 
\newenvironment{theorem}[2][Theorem]{\begin{trivlist}
\item[\hskip \labelsep {\bfseries #1}\hskip \labelsep {\bfseries #2.}]}{\end{trivlist}}
\newenvironment{lemma}[2][Lemma]{\begin{trivlist}
\item[\hskip \labelsep {\bfseries #1}\hskip \labelsep {\bfseries #2.}]}{\end{trivlist}}
\newenvironment{exercise}[2][Exercise]{\begin{trivlist}
\item[\hskip \labelsep {\bfseries #1}\hskip \labelsep {\bfseries #2.}]}{\end{trivlist}}
\newenvironment{problem}[2][Problem]{\begin{trivlist}
\item[\hskip \labelsep {\bfseries #1}\hskip \labelsep {\bfseries #2.}]}{\end{trivlist}}
\newenvironment{question}[2][Question]{\begin{trivlist}
\item[\hskip \labelsep {\bfseries #1}\hskip \labelsep {\bfseries #2.}]}{\end{trivlist}}
\newenvironment{corollary}[2][Corollary]{\begin{trivlist}
\item[\hskip \labelsep {\bfseries #1}\hskip \labelsep {\bfseries #2.}]}{\end{trivlist}}

\newenvironment{solution}{\begin{proof}[Solution]}{\end{proof}}
 
\begin{document}
\pagestyle{fancy}
\fancyhf{}
\fancyhf[EHC]{Davis McKenzie - Ionospheric Radio Project 1}
\fancyhf[EFC]{\thepage}
\fancyhf[OFC]{\thepage}
\fancyhf[OHC]{Davis McKenzie - Ionospheric Radio Project 1}
% --------------------------------------------------------------
%                         Start here
% --------------------------------------------------------------
 
\title{Project 1, Atmospheric Ray Tracing}
\author{Davis McKenzie\\ %replace with your name
Ionospheric Radio}

\maketitle

\noindent The following report prepared for ELEC 6970 - Special Topics: Ionospheric Radio is Structured as follows:

\begin{enumerate} %the first video is required (but you don't have to watch it first), you can choose the other 4 from the list on the Course Materials page
\item Project Motivations 
\item Two-Dimensional Ray Tracing Implementation and Results
\item Three-Dimensional Ray Tracing Implementation and Results
\item Further Uses of PHaRLAP and Ray Tracing
\end{enumerate}



\newpage

\section{Project Motivations}
The ionosphere plays a large role in long-distance radio propagation, especially in the high-frequency (HF) range. Using the PHaRLAP toolbox available from the Defence Science and Technology Group of the Australian Department of Defence \cite{Cervera_2024}, the behavior of individual propagating rays may be studied. Apart from studying HF propagation, this tool can also be used to learn about how real-world systems, such as ionosondes, make measurements. By extracting useful information such as the maximum height reached by rays and their path data, characteristics of the ionosphere may be deduced in a manner similar to how functional instruments make these determinations.



% --------------------------------------------------------------
%     You don't have to mess with anything below this line.
% --------------------------------------------------------------

\section{Two Dimensional Ray Tracing Implementation and Results}
\subsection{Block Diagram}
Shown below is the block diagram outlining the general construction of the signal generator. Subsystems are shown below in the block diagram and are also highlighted in the final schematic.
\begin{figure}[h!] 
\centering 
\includegraphics[width=4.5in]{img/Screenshot 2024-12-04 030231.png} 
\label{fig:myPhoto1} 
\end{figure}
\subsection{Astable Multivibrator}
At the heart of this signal generator lies this op-amp-based astable multivibrator. This is where the output frequency is set, and a square wave is generated to be passed to further subsystems. The following equation governs the function of the multivibrator.

\begin{equation*}
    T =  2\cdot R_3\cdot C_3\cdot  ln\left(\frac{1+\beta}{1-\beta}\right)
\end{equation*}
Where T is the period and $\beta$ is:
\begin{equation*}
    \frac{R_2}{R_1 + R_2}
\end{equation*}
To begin, T was calculated by inverting the desired frequency. $R_1$ and $R_2$ were then chosen as equal values with values conducive to efficient op-amp function. Substituting and simplifying, the following equation is reached.
\begin{equation*}
    RC = \frac{0.0001}{ln(3)}
\end{equation*}
\newpage
This leaves on R and c as unknowns in the circuit. C was then picked as a round value which would lead to R lying within the list of acceptable components. The final components selected for the astable multivibrator are as follows:
\begin{equation*}
    R_1 = R_2 = 12\,k\Omega
\end{equation*}
\begin{equation*}
    R_3 = 7.5\,k\Omega
\end{equation*}
\begin{equation*}
    C_3 = 20\,nF
\end{equation*}

\subsection{Integrator}
The next step in producing a sinusoidal output waveform for the generator is by transforming the square wave generated by the multivibrator into a sawtooth wave. This can be achieved by passing the signal to an integrator. Suitable values for R and C were chosen to maintain a suitable output amplitude as governed by the equation below.
\begin{equation*}
    V_{out} = -\frac{1}{R_{in}\cdot C}\int^{t}_{0} V_{in} dt
\end{equation*}
The chosen values are as follows
\begin{equation*}
    R_4 = 100\,k\Omega
\end{equation*}
\begin{equation*}
    C_4 = 0.5\,nF
\end{equation*}

\subsection{Low-Pass Filter}
The low-pass filter is the final step in forming the sinusoidal output of the generator. The filter selects the fundamental frequency of the sawtooth wave and outputs a sinusoid matching this frequency. The components were selected to abide by the following equations:
\begin{equation*}
    A_v(s) = \frac{R_6}{R_5}\cdot \frac{1}{(1+sR_2C)} = \frac{R_6}{R_5}\cdot \frac{1}{(1+\frac{s}{\omega_H})}
\end{equation*}
where
\begin{equation*}
    \omega_H = 2\pi f_h = \frac{1}{R_2C}
\end{equation*}
Using these equations and a suitable frequency, the following component values were selected.
\begin{equation*}
    R_5 = 9.31\,k\Omega \;OR \;619\,\Omega
\end{equation*}
\begin{equation*}
    R_6 = 20\,k\Omega
\end{equation*}
\begin{equation*}
    C_6 = 2\,nF
\end{equation*}

\newpage
\section{Three-Dimensional Ray Tracing Implementation and Results}
Each component subsystem was then combined and the resulting completed signal generator is shown below.

\begin{figure}[h!] 
\centering 
\includegraphics[width=6.5in]{img/Signal Generator Schematic.png} 
\label{fig:myPhoto1} 
\end{figure}

A key component of the function generator is its ability to output a variable output amplitude. The selected output amplitudes (0.1 V and 1 V) have their output waveforms shown below.

\begin{figure}[h!] 
\centering 
\includegraphics[width=6in]{img/Screenshot 2024-12-04 045412.png} 
\label{fig:myPhoto1} 
\end{figure}
\newpage
\begin{figure}[h!] 
\centering 
\includegraphics[width=6in]{img/Screenshot 2024-12-04 045341.png} 
\label{fig:myPhoto1} 
\end{figure}

These output waveforms appear to be very pure sinusoids, but there is no way to visually confirm this. Therefore, an FFT was performed on the output. The resulting Plot is shown below.
\begin{figure}[h!] 
\centering 
\includegraphics[width=6in]{img/Screenshot 2024-12-04 045933.png} 
\label{fig:myPhoto1} 
\end{figure}

The output of the signal generator appears to be a relatively pure sinusoid according to the FFT, confirming the visual analysis. The FFT also shows that the signal is being output at the correct frequency of 5 kHz.
\\ 

Overall, the output of the signal generator is clean and accurate. However, there are minor discrepancies in the expected vs. observed amplitudes. This could be improved in revisions of this design.

\section{Further Uses of PHaRLAP and Ray Tracing}



\printbibliography
\end{document}
