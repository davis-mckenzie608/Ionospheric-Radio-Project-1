    % --------------------------------------------------------------
% This is all preamble stuff that you don't have to worry about.
% Head down to where it says "Start here"
% --------------------------------------------------------------
 
\documentclass[12pt]{article}
 
\usepackage[margin=1in]{geometry} 
\usepackage{amsmath,amsthm,amssymb}
\usepackage{fancyhdr}
\usepackage{graphicx}
\usepackage[
backend=biber,
style=ieee,
]{biblatex}
\addbibresource{bibliography.bib}
 
\newcommand{\N}{\mathbb{N}}
\newcommand{\Z}{\mathbb{Z}}
 
\newenvironment{theorem}[2][Theorem]{\begin{trivlist}
\item[\hskip \labelsep {\bfseries #1}\hskip \labelsep {\bfseries #2.}]}{\end{trivlist}}
\newenvironment{lemma}[2][Lemma]{\begin{trivlist}
\item[\hskip \labelsep {\bfseries #1}\hskip \labelsep {\bfseries #2.}]}{\end{trivlist}}
\newenvironment{exercise}[2][Exercise]{\begin{trivlist}
\item[\hskip \labelsep {\bfseries #1}\hskip \labelsep {\bfseries #2.}]}{\end{trivlist}}
\newenvironment{problem}[2][Problem]{\begin{trivlist}
\item[\hskip \labelsep {\bfseries #1}\hskip \labelsep {\bfseries #2.}]}{\end{trivlist}}
\newenvironment{question}[2][Question]{\begin{trivlist}
\item[\hskip \labelsep {\bfseries #1}\hskip \labelsep {\bfseries #2.}]}{\end{trivlist}}
\newenvironment{corollary}[2][Corollary]{\begin{trivlist}
\item[\hskip \labelsep {\bfseries #1}\hskip \labelsep {\bfseries #2.}]}{\end{trivlist}}

\newenvironment{solution}{\begin{proof}[Solution]}{\end{proof}}
 
\begin{document}
\pagestyle{fancy}
\fancyhf{}
\fancyhf[EHC]{Davis McKenzie - Ionospheric Radio Project 1}
\fancyhf[EFC]{\thepage}
\fancyhf[OFC]{\thepage}
\fancyhf[OHC]{Davis McKenzie - Ionospheric Radio Project 1}
% --------------------------------------------------------------
%                         Start here
% --------------------------------------------------------------
 
\title{Project 1, Atmospheric Ray Tracing}
\author{Davis McKenzie\\ %replace with your name
Ionospheric Radio}

\maketitle

\noindent The following report prepared for ELEC 6970 - Special Topics: Ionospheric Radio is Structured as follows:

\begin{enumerate} %the first video is required (but you don't have to watch it first), you can choose the other 4 from the list on the Course Materials page
\item Project Motivations 
\item Two-Dimensional Ray Tracing Implementation and Results
\item Three-Dimensional Ray Tracing Implementation and Results
\item Further Uses of PHaRLAP and Ray Tracing
\end{enumerate}



\newpage

\section{Project Motivations}
The ionosphere plays a large role in long-distance radio propagation, especially in the high-frequency (HF) range. Using the PHaRLAP toolbox available from the Defence Science and Technology Group of the Australian Department of Defence \cite{Cervera_2024}, the behavior of individual propagating rays may be studied. Apart from studying HF propagation, this tool can also be used to learn about how real-world systems, such as ionosondes, make measurements. By extracting useful information such as the maximum height reached by rays and their path data, characteristics of the ionosphere may be deduced in a manner similar to how functional instruments make these determinations. For this project, oblique ionograms are generated by analyzing rays traced from sites in Chesapeake, Virginia, USA and Corpus Christi, Texas, USA.


\section{Two Dimensional Ray Tracing Implementation and Results}
The \verb*|raytrace_2d()| function is available in PHaRLAP to trace rays in a plane located between two points on Earth's surface. An example of the rays traced by this function is shown below in Figure 1. \verb*|raytrace_2d()| was used to generate rays for further analysis in this section.
\begin{figure}[h!] 
	\centering 
	\includegraphics[width=6.5in]{img/Raytrace_ex.png} 
	\label{fig:2d_Ray_Ex} 
	\caption{An Example of Rays Traced by the 2D Ray-Tracing function in PHaRLAP.}
\end{figure}
\subsection{Generation of Rays and Ray Data}
To generate all necessary inputs for the ray tracing function, a function to calculate the bearing and distance between two coordinate points and Earth was created, \verb*|Bearing_Calculator|. This function takes two points given in degrees, minutes, seconds format using the WGS84 
\subsection{Filtering of Rays to Valid Paths}
Shown below is the block diagram outlining the general construction of the signal generator. Subsystems are shown below in the block diagram and are also highlighted in the final schematic.
\subsection{Extraction of Data from Valid Rays}
At the heart of this signal generator lies this op-amp-based astable multivibrator. This is where the output frequency is set, and a square wave is generated to be passed to further subsystems. The following equation governs the function of the multivibrator.
\subsection{Calculation of Virtual Reflection Height}
The next step in producing a sinusoidal output waveform for the generator is by transforming the square wave generated by the multivibrator into a sawtooth wave. This can be achieved by passing the signal to an integrator. Suitable values for R and C were chosen to maintain a suitable output amplitude as governed by the equation below.
\subsection{Results and Further Processing}
The low-pass filter is the final step in forming the sinusoidal output of the generator. The filter selects the fundamental frequency of the sawtooth wave and outputs a sinusoid matching this frequency. The components were selected to abide by the following equations:

\section{Three-Dimensional Ray Tracing Implementation and Results}


\section{Further Uses of PHaRLAP and Ray Tracing}



\printbibliography
\end{document}
